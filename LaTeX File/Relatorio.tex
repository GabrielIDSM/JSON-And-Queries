\documentclass[a4paper, 12pt] {article}

\title{Relatório}
\author{Gabriel Inácio\\gabriel.inacio@telcomanager.com}
\date{12/2020}

\usepackage[top=2cm, bottom=2cm, right=2.5cm, left=2.5cm]{geometry}
\usepackage[utf8]{inputenc}
\usepackage[brazil]{babel}
\usepackage{amsmath, amsfonts, amssymb, indentfirst, gensymb, graphicx, float}

\usepackage{xcolor}
\usepackage{listings}

\definecolor{mGreen}{rgb}{0,0.6,0}
\definecolor{mGray}{rgb}{0.5,0.5,0.5}
\definecolor{mPurple}{rgb}{0.58,0,0.82}
\definecolor{backgroundColour}{rgb}{0.95,0.95,0.92}

\lstdefinestyle{CStyle}{
    backgroundcolor=\color{backgroundColour},   
    commentstyle=\color{mGreen},
    keywordstyle=\color{magenta},
    numberstyle=\tiny\color{mGray},
    stringstyle=\color{mPurple},
    basicstyle=\footnotesize,
    breakatwhitespace=false,         
    breaklines=true,                 
    captionpos=b,                    
    keepspaces=true,                 
    numbers=left,                    
    numbersep=5pt,                  
    showspaces=false,                
    showstringspaces=false,
    showtabs=false,                  
    tabsize=2,
    language=C
}

\colorlet{punct}{red!60!black}
\definecolor{background}{rgb}{0.95,0.95,0.92}
\definecolor{delim}{RGB}{20,105,176}
\colorlet{numb}{magenta!60!black}

\lstdefinelanguage{json}{
    basicstyle=\normalfont\ttfamily,
    numbers=left,
    numberstyle=\scriptsize,
    stepnumber=1,
    numbersep=8pt,
    showstringspaces=false,
    breaklines=true,
    frame=lines,
    backgroundcolor=\color{background},
    literate=
     *{0}{{{\color{numb}0}}}{1}
      {1}{{{\color{numb}1}}}{1}
      {2}{{{\color{numb}2}}}{1}
      {3}{{{\color{numb}3}}}{1}
      {4}{{{\color{numb}4}}}{1}
      {5}{{{\color{numb}5}}}{1}
      {6}{{{\color{numb}6}}}{1}
      {7}{{{\color{numb}7}}}{1}
      {8}{{{\color{numb}8}}}{1}
      {9}{{{\color{numb}9}}}{1}
      {:}{{{\color{punct}{:}}}}{1}
      {,}{{{\color{punct}{,}}}}{1}
      {\{}{{{\color{delim}{\{}}}}{1}
      {\}}{{{\color{delim}{\}}}}}{1}
      {[}{{{\color{delim}{[}}}}{1}
      {]}{{{\color{delim}{]}}}}{1},
}

\begin{document}
	\maketitle \newpage \tableofcontents \newpage
	\section{C/C++}
		\subsection{jq - Command-line JSON processor}
			\subsubsection{Introdução}
				O projeto jq o define como \textit{“Um processador de linha de comando para arquivos JSON leve e flexível” }. Ele foi desenvolvido com o intuito de ser como o \textit{sed} para arquivos JSON. 

				É um projeto de código aberto, com código disponível no GitHub (\textit{https://github.com/stedolan/jq}).
			\subsubsection{Sintaxe}
				Considerando um contexto ideal de uma biblioteca que recebe um JSON com uma query e retorna um JSON, o jq consegue realizar isso por meio de \textbf{filtros}. Os filtros não são tão avançados quanto uma query SQL, porém conseguem entregar um conjunto de informações restrita que pode ser utilizada.
			\subsubsection{Filtros Básicos}
				\textbf{Identidade}

\begin{lstlisting}[language=bash]
$  jq '.'
\end{lstlisting}

				\textbf{Identificador de Objeto-Índice}

\begin{lstlisting}[language=bash]
$  jq '.foo'
Input	{"foo": 42, "bar": "less interesting data"}
Output 	42

$ jq '.foo'
Input	{"notfoo": true, "alsonotfoo": false}
Output 	null
\end{lstlisting}

				\newpage \textbf{Identificador de Objeto-Índice Opcional}

\begin{lstlisting}[language=bash]
$ jq '.foo?'
Input	{"foo": 42, "bar": "less interesting data"}
Output 	42
	
$ jq '.foo?'
Input	{"notfoo": true, "alsonotfoo": false}
Output 	null

\end{lstlisting}

				\textbf{Arrays}

\begin{lstlisting}[language=bash]
$ jq '.[0]'
Input	[{"name":"JSON", "good":true}, {"name":"XML", "good":false}]
Output 	{"name":"JSON", "good":true}
	
$ jq '.[2]'
Input	[{"name":"JSON", "good":true}, {"name":"XML", "good":false}]
Output 	null
\end{lstlisting}

				\textbf{Slice}

\begin{lstlisting}[language=bash]
$ jq '.[2:4]'
Input	["a","b","c","d","e"]
Output 	["c", "d"]
	
$ jq '.[2:4]'
Input	"abcdefghi"
Output 	"cd"
	
$ jq '.[:3]'
Input	["a","b","c","d","e"]
Output 	["a", "b", "c"]
\end{lstlisting}

				\textbf{Vírgula}

\begin{lstlisting}[language=bash]
$ jq '.foo, .bar'
Input	{"foo": 42, "bar": "something else", "baz": true}
Output 	42
	"something else"
	
$ jq '.user, .projects[]'
Input	{"user":"stedolan", "projects": ["jq", "wikiflow"]}
Output 	"stedolan"
	"jq"
	"wikiflow"
\end{lstlisting}

				\textbf{Barra Vertical (Pipe)}

\begin{lstlisting}[language=bash]
jq '.[] | .name'
Input	[{"name":"JSON", "good":true}, {"name":"XML", "good":false}]
Output 	"JSON"
	"XML"
\end{lstlisting}

				\noindent A documentação completa da ferramenta pode ser acessada em \textit{https://stedolan.github.io/jq/manual}.
			\newpage \subsubsection{Exemplo}
				Considerando o mesmo arquivo “Request.json", podemos obter, por exemplo, o atributo “params" de cada um dos objetos presentes no array “report":
\begin{lstlisting}[language=bash]
$ jq '.reports|.[]|.params' Request.json
\end{lstlisting}

				O resultado desse comando será:

\begin{lstlisting}[language=json,firstnumber=1]
{
  "params": 0,
  "params": 6
}
\end{lstlisting}

				O filtro “.[]" tem a função de percorrer todos os elementos do array “reports".
		\subsection{jRead - An in-place JSON Element Reader}
			\subsubsection{Introdução}
			\subsubsection{Sintaxe}
			\subsubsection{Exemplo}
\end{document}


















